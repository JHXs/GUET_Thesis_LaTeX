%
%  =======================================================================
%  ····Y88b···d88P················888b·····d888·d8b·······················
%  ·····Y88b·d88P·················8888b···d8888·Y8P·······················
%  ······Y88o88P··················88888b·d88888···························
%  ·······Y888P··8888b···88888b···888Y88888P888·888·88888b·····d88b·······
%  ········888······"88b·888·"88b·888·Y888P·888·888·888·"88b·d88P"88b·····
%  ········888···d888888·888··888·888··Y8P··888·888·888··888·888··888·····
%  ········888··888··888·888··888·888···"···888·888·888··888·Y88b·888·····
%  ········888··"Y888888·888··888·888·······888·888·888··888··"Y88888·····
%  ·······························································888·····
%  ··························································Y8b·d88P·····
%  ···························································"Y88P"······
%  =======================================================================
% 
%  -----------------------------------------------------------------------
% Author       : 焱铭
% Date         : 2024-05-31 22:39:58 +0800
% LastEditTime : 2025-02-07 14:27:36 +0800
% Github       : https://github.com/YanMing-lxb/
% FilePath     : /GUET_Thesis_LaTeX/Chapters/Chapter4.tex
% Description  : 
%  -----------------------------------------------------------------------
%

% !Mode:: "TeX:UTF-8"

\chapter{实验结果与分析}\label{ch:4}

本章在\cref{ch:3}完成基……。

写作目标:用数据说话,证明你的模型比别人的好(SOTA)。

\section{数据集介绍与预处理(来源、统计特征、划分方式)}
比如 UCI 数据集、中国空气质量在线监测平台数据。列出时间跨度、采样频率。


\subsection{数据集来源与统计特征}
[核心图表] 贴热力图(展示空间相关性)、贴时序折线图(展示季节性/周期性)。


\subsection{数据的时空相关性分析(EDA,探索性数据分析)}


\subsection{数据预处理(归一化、去噪等)}

\section{评价指标}

列出公式:MAE, RMSE, MAPE, R2

\section{对比模型与实验环境设置}

基线模型:LSTM, GRU, LSTNet, Transformer, Informer, PatchTST(原版)。
环境:Python 版本, PyTorch 版本, 显卡型号(如 RTX 3090)。

\section{ 模型对比实验分析(总体性能表)}

[核心图表] 一个大表格,横向是模型,纵向是指标。你的模型数据要加粗表示最优。
文字分析:为什么你的比 LSTM 好?(有 Attention);为什么比 Informer 好?(因为有 Patching 和空间聚合)。

\section{不同预测步长的性能分析}

对比预测未来 1h, 6h, 12h, 24h 的误差。画柱状图或折线图。

结论:证明模型在长短期预测上都稳定。

\section{消融实验}
设计变体:① 去掉空间聚合模块;② 去掉 Patching 机制;③ 去掉通道独立。

结论:证明每个模块都是有用的,缺一不可。

\section{参数敏感性分析(如 Patch Size, Look-back Window 长度影响)}

\section{本章小节}