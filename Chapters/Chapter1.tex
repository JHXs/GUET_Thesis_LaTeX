%
%  =======================================================================
%  ····Y88b···d88P················888b·····d888·d8b·······················
%  ·····Y88b·d88P·················8888b···d8888·Y8P·······················
%  ······Y88o88P··················88888b·d88888···························
%  ·······Y888P··8888b···88888b···888Y88888P888·888·88888b·····d88b·······
%  ········888······"88b·888·"88b·888·Y888P·888·888·888·"88b·d88P"88b·····
%  ········888···d888888·888··888·888··Y8P··888·888·888··888·888··888·····
%  ········888··888··888·888··888·888···"···888·888·888··888·Y88b·888·····
%  ········888··"Y888888·888··888·888·······888·888·888··888··"Y88888·····
%  ·······························································888·····
%  ··························································Y8b·d88P·····
%  ···························································"Y88P"······
%  =======================================================================
% 
%  -----------------------------------------------------------------------
% Author       : 焱铭
% Date         : 2024-04-17 20:49:49 +0800
% LastEditTime : 2025-10-26 15:42:27 +0800
% Github       : https://github.com/YanMing-lxb/
% FilePath     : /GUET_Thesis_LaTeX/Chapters/Chapter1.tex
% Description  : 
%  -----------------------------------------------------------------------
%

% !Mode:: "TeX:UTF-8"
%此为第一章节。
%[h]为hear代码所在位置,\caption为表注题注,\cref{}引用图表公式章节等,\cite为引用参考文献,\subfloat子图,\label标签,\begin{figure}图片环境,\begin{table}表格环境,\begin{equation}公式环境,\toprule三线表顶线,\cmidrule三线表中线,\bottomrule三线表底线,\begin{theorem}定理,\begin{proof}证明,\begin{corollary}推论,\begin{lemma}引理
    
\chapter{绪论}\label{ch:1}

写作目标:把故事讲圆。告诉读者为什么要做这个,别人做得怎么样,你打算怎么做。

\section{研究背景和意义}
**背景**:空气污染的危害(健康、经济);现有监测站点的局限性;大数据时代数据驱动方法的兴起。

**意义**:精准预测对政府决策(限行、工厂停工)和个人出行(佩戴口罩)的指导意义。

\section{国内外研究现状}
请详细阅读本项目根目录下的README.md 文档

\section{研究内容}


\section{本论文的结构安排}
\cref{ch:1}:绪论。本章主要进行整体说明。

\cref{ch:2}:图片示例。

\cref{ch:3}:表格示例。

\cref{ch:4}:数学公式示例。

\cref{ch:5}:列表、算法、定理、证明插入示例。

\cref{ch:6}:全文总结与展望。本次研究工作进行总结,并根据全文研究过程中……。


