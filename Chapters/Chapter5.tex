%
%  =======================================================================
%  ····Y88b···d88P················888b·····d888·d8b·······················
%  ·····Y88b·d88P·················8888b···d8888·Y8P·······················
%  ······Y88o88P··················88888b·d88888···························
%  ·······Y888P··8888b···88888b···888Y88888P888·888·88888b·····d88b·······
%  ········888······"88b·888·"88b·888·Y888P·888·888·888·"88b·d88P"88b·····
%  ········888···d888888·888··888·888··Y8P··888·888·888··888·888··888·····
%  ········888··888··888·888··888·888···"···888·888·888··888·Y88b·888·····
%  ········888··"Y888888·888··888·888·······888·888·888··888··"Y88888·····
%  ·······························································888·····
%  ··························································Y8b·d88P·····
%  ···························································"Y88P"······
%  =======================================================================
% 
%  -----------------------------------------------------------------------
% Author       : 焱铭
% Date         : 2023-12-03 15:43:39 +0800
% LastEditTime : 2025-01-14 20:54:49 +0800
% Github       : https://github.com/YanMing-lxb/
% FilePath     : /GUET_Thesis_LaTeX/Chapters/Chapter5.tex
% Description  : 
%  -----------------------------------------------------------------------
%

% !Mode:: "TeX:UTF-8"

\chapter{多元空气质量数据预测系统的设计与实现}\label{ch:5}

写作目标:展示工程能力,表明你的研究不仅仅是纸上谈兵。

\section{系统需求分析}

功能需求:数据查询、模型预测、历史记录查看。
非功能需求:响应速度、易用性。


\section{系统总体架构设计(前后端分离、模型部署架构)}

架构图:B/S 架构。

前端:Vue.js / React + ECharts (可视化)。

后端:SpringBoot / Flask / Django。

AI服务:模型加载(ONNX / PyTorch),提供 API 接口。

\section{系统数据库设计}

E-R 图。

主要表结构:用户表、站点信息表、空气质量数据表、预测结果表

\section{核心功能模块设计与实现}

贴关键代码片段(如后端调用模型的代码,前端绘制图表的代码)。

重点:描述数据是如何从数据库流向模型,再返回给前端的。

\section{系统测试与界面展示}


[多图预警] 贴系统截图:

登录界面。

实时监测大屏(地图+数据)。

预测分析页面(折线图对比真实值与预测值)。

\section{本章小结}