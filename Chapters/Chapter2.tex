% !Mode:: "TeX:UTF-8"
%此为章节二模板
%\chapter、\section、\subsection、\subsubsection分别对应一二三四级标题
% file: Chapters/Chapter2.tex
\chapter{时间序列预测相关理论}\label{ch:2}
时间序列即指按照时间顺序排列的一组观测值,其往往具有一定的趋势性、周期性、随机性等特点。我们生活的方方面面都包含着时间序列,例如智能手表的心率监测数据、股票的 Tick 数据、外卖 App 的实时订单量、微信运动每日步数、风力发电机的实时发电功率等等都属时间序列,当然本研究涉及的空气质量数据也是一种时间序列数据。时间序列按连续性分类可以分为连续时间序列和离散时间序列,离散时间序列可以理解为在特定时间点或者间隔多少时间观测一次数据,比如每日股票的收盘价、每月GDP数据等等,连续时间序列则是随时间连续记录数据,例如实时气温监测数据、服务器的实时负载信息。时间序列按特征数量分类可分为单变量时间序列、多元时间序列和高维时间序列,单变量时间序列即只有观测一个特征变量,这个时间序列数据只包含了一个特征变量,比如心率监测数据它只包含心率这一个特征变量,它就是一个单变量时间序列,再比如微信运动每日步数数据只包含步数这一个特征变量,那么它也是一个单变量时间序列。多元时间序列也就是在同一时间轴上同时观测两个或多个特征变量,不同特征变量之间往往会相互影响,比如本研究设计的空气质量监测数据,它往往包含有CO、NO、O3、PM10、PM2.5、SO2、温度、湿度等等多种特征变量,它就是一个典型的多元时间序列。还有像工业IoT中多个传感器设备检测不同的变量,电网多个传感器监测负荷、电价等不同电气量,交通监测路段流量、速度、占用率等都是属于多元时间序列。

\section{时间序列预测方法概述}
对于时间序列的预测有单步预测、多步预测,还有确定性预测以及概率性预测。单步预测一次预测只输出未来的一个时间点的预测值,例如给定真实历史数据 $x_1,x_2,\dots,x_t$,预测 $t+1$ 时刻的值 $\hat{y}_{t+1}$。而多步预测则是模型一次性预测多个未来时间点的值,同样给定真实历史数据 $x_1,x_2,\dots,x_t$,模型一次性预测未来 h 个时间点的值 \(\hat{y}_{t+1},\dots,\hat{y}_{t+h}\),其中 $h \geq 2$ 为预测步长。对于单步预测模型,其想要实现预测多个未来时间值,可以通过递归的方式实现,就是说其还是一次预测未来一个时间步的值,但是通过将这个预测的时间值作为历史数据再次进行预测下一个时间步的值,直到运行 h 次得到 h 个未来时间的值。但是这样做由于使用了预测值进行预测,而每次预测都会有误差的存在,因此,随着预测时间步数的增长,势必会由于误差的累计导致多步预测的效果越来越差,因此这种通过单步预测模型递归实现多步预测的方式不是一种准确高效的方式,如果需要进行更长期的预测,构建多步预测模型是更好的选择。单步预测和多步预测详细图如\cref{fig:single_multi_step_forecast}所示。

\begin{figure}[htb]
    \centering
    \includegraphics[width=0.8 \textwidth]{single_multi_step_forecast.drawio.png}
    \caption[单步预测与多步预测示意图]{单步预测与多步预测示意图} % 中括号中内容为插图索引中显示内容,可在题注内容过长时使用
    \label{fig:single_multi_step_forecast} % 用于在正文中引用该图片,如图~\ref{fig:single_multi_step_forecast}
\end{figure}


对于本研究涉及的空气质量数据预测,其具有自己的特殊性。空气质量数据一般是多元的,其一般涉及多种不同的变量,例如 PM2.5、PM10、O3、碳化物、氮化物、硫化物等等。同时其中一个空气质量监测站的数据往往也同时跟其他一些空气质量监测站有着强关联性,比如目标站点附近的站点的空气质量数据往往更目标站点有着很强的相关性,然后同时,在目标站点上风向的监测站点数据往往对于目标站点也有很强的影响,可能表现为目标站点一个特征变量的时间序列数据是上风向一些站点的延时序列。因此对于空气质量数据,只考虑一个目标站点的数据往往是不够的,综合考虑多个站点的数据通常可以获得更多有用的信息以提高预测的性能。空气质量预测同时也受其他因素的强影响,具体的气象因素就会对空气质量产生很大的影响,比如天气情况、风速风向等,下雨天由于雨水的对颗粒物的附着以及有一些会溶解到水中,因此很多污染物便随着雨水一起沉降到了地面,空气质量也会有所提升,在雨后人们总能够感觉空气是清新的。如果一个地区位于空气污染排放地区的下风向,那么这个地方的空气质量通常会跟上风向的排放息息相关,同时风速也会影响空气质量,风力越大,污染物的扩散更加迅速,范围更大,从而起到稀释作用,空气质量会更加好一些。因此不同的气象因素与污染物浓度之间有着复杂的非线性关系,另一方面,不同污染物之间也可能产生不同的化学反应进一步加强了非线性性,增加了空气质量预测任务的挑战。

综上所述,空气质量预测任务不仅涉及多元变量之间的耦合关系,还受到气象条件和污染物化学反应的影响,表现出显著的非线性特征。这种复杂性决定了单一的线性模型难以满足预测需求。为此,研究者们在不同阶段提出了多种预测方法:早期主要依赖传统统计方法,如 ARIMA、SARIMA 等,它们在处理平稳单变量序列时具有一定优势;随后,机器学习方法逐渐兴起,能够突破线性假设,更好地刻画非线性关系;近年来,随着数据规模和计算能力的提升,深度学习方法成为主流,凭借自动特征提取和强大的时序建模能力,在空气质量预测中展现出独特优势。下面将依次介绍这些方法。

\subsection{传统统计方法}

从时间序列预测方法的发展历程来看,最早期的时间序列预测方法主要是基于传统的统计学方法,这些方法通常假设时间序列数据是平稳的,并且主要关注数据的线性特征。常见的传统统计方法包括自回归模型(AR)、移动平均模型(MA)、自回归移动平均模型(ARMA)以及自回归积分滑动平均模型(ARIMA)等。这些方法通过对历史数据进行建模,捕捉时间序列中的趋势和季节性变化,实现对未来数据的预测。

自回归模型(Autoregressive Model, AR)是常见的时间序列预测模型,它通过线性组合过去的观测值来预测当前值。其数学表达式为 \(X_t = c + \sum_{i=1}^{p} \varphi_i X_{t-i} + \varepsilon_t\)

然而,传统统计方法在处理非线性关系和多变量时间序列时存在一定的局限性,难以充分挖掘数据中的复杂模式。

\subsection{机器学习方法}

\section{深度学习基础理论}

\subsection{神经网络基础}

\subsection{序列模型}

\subsection{注意力机制}

\section{Transformer 模型原理}


\section{本章小节}
本章介绍了